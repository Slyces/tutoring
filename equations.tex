\documentclass{article}
% ---------------------------------------------------------------------------- %
\usepackage{style}

% ---------------------------------------------------------------------------- %
% Margins
\usepackage[top=2.5cm, bottom=2.5cm, left=2cm, right=2cm]{geometry}

% ---------------------------------------------------------------------------- %
% Page de garde
\title{%
    {\huge\textsc{Tea Time Tutorials}} \\
    Simple Equations
}
\author{
    Simon Lassourreuille
}
\date{}

\begin{document}

\maketitle

% ──────────────────────────────────────────────────────────────────────────── %
%                                                                              %
% ──────────────────────────────────────────────────────────────────────────── %
\section{Introduction to equations}

\begin{minipage}{0.475\textwidth}
    An equation is a mathematical expression that shows that two things are equal. The most important part of an equation is the << \kw{$=$} >>. You can see on the right an example of the format of an equation.

\end{minipage}
\hfill\vline\hfill
\begin{minipage}{0.475\textwidth}
    {\center \textsc{Format of an equation}}
    \begin{boxequation}
        \begin{equation}
            \righteq \kw{\ =\ } \lefteq
        \end{equation}
    \end{boxequation}
\end{minipage}

\subsection{What can I do with an equation ?}

An equation is just a statement saying that \lefteq\ and \righteq\ have the same value. We are allowed to do some transformations on an equation without changing its veracity.

\noindent There is a total of 5 authorized transformations:

\begin{minipage}{0.475\textwidth}
    \textsc{Swapping both sides}
\end{minipage}
\hfill\vline\hfill
\begin{minipage}{0.475\textwidth}
    \begin{boxequation}
        \setcounter{equation}{0}
        \begin{align}
            \defpoint{A}{\righteq}\kw{\ &=\ } \lefteq \\
            \defpoint{B}{\lefteq}\kw{\ &=\ } \righteq
            \steparrow{A}{B}
        \end{align}
    \end{boxequation}
\end{minipage}



\end{document}
